\documentclass[12pt]{article}%
\usepackage[english]{babel}
\usepackage[utf8x]{inputenc}
\usepackage{amsfonts}
\usepackage{fancyhdr}
\usepackage{comment}
\usepackage[a4paper, top=2.5cm, bottom=2.5cm, left=2.2cm, right=2.2cm]%
{geometry}
\usepackage{times}
\usepackage{amsmath}
\usepackage{changepage}
\usepackage{amssymb}

\begin{document}

\title{Seguridad informática - Entrega 1}
\author{Lauro J. Figueroa}
\date{\today}
\maketitle
\section*{Introducción}
Los ejercicios realizados fueron: 3, 

\section*{Ejercicio 3}
El principal objetivo de una entidad comercial es producir y es
contraproducente, al menos intuitivamente, imponer barreras para
maximizar este objetivo. Sin embargo, hay entidades que no podrian
existir o al menos no como las conocemos hoy sin políticas de seguridad y
modelos robustos para garantizar que solo los usuarios con la
autorización necesaria puedan operar en ellos.

Algunos ejemplos de estas entidades y sus políticas son las siguientes:
\begin{itemize}
\item Bancos y su servicios de homebanking, donde una de sus políticas
  de seguridad es usualmente desloguear deliberadamente a los usarios
  loguados después de cierto tiempo. Los modelos que implementan esta
  política, suelen definir un tiempo de diez minutos.
  
\item Plataformas comerciales de compra, como Steam. Esta empresa
  tiene como caso interesante de estudio como la política de seguridad
  para establecer un canal de comunicación con un dispositibo
  confiable. Donde el modelado de esta política ofrece al usuario
  puede designar a un dispositivo como confiable.

  Una vez que se instala y configura la aplicación oficial en algun
  dispositivo soportado, ésta es usada para verificar que quién
  intenta loguearse a la cuenta del usuario es el mismisimo usuario y
  no alguien más. La autenticación del usuario se da traves de la
  generación de un código único enviado al dispositivo de confianza.
  Cada compra echa en la aplicación tambien usa este mecanismo.

  Entonces es claro que en el caso que el usuario pierda, o
  simplemente, no pueda acceder a su dispositivo no podrá loguearse ni
  relizar comprá alguna en esta plataforma, representando así una
  potencial barrera comercial.
\end{itemize}




\end{document}