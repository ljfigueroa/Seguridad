\documentclass[12pt]{article}%
\usepackage[english]{babel}
\usepackage[utf8x]{inputenc}
\usepackage{amsfonts}
\usepackage{fancyhdr}
\usepackage{comment}
\usepackage[a4paper, top=2.5cm, bottom=2.5cm, left=2.2cm, right=2.2cm]%
{geometry}
\usepackage{times}
\usepackage{amsmath}
\usepackage{changepage}
\usepackage{amssymb}

\begin{document}

\title{Seguridad informática - Entrega 1}
\author{Lauro J. Figueroa}
\date{Septiembre 29, 2017}
\maketitle
\section*{Introducción}
Los ejercicios realizados fueron: 3, 7, 11, 15, 19.5 y 21.3.

\section*{Ejercicio 3}
El principal objetivo de una entidad comercial es producir y es
contraproducente, al menos intuitivamente, imponer barreras para
maximizar este objetivo. Sin embargo, hay entidades que no podrían
existir o al menos no como las conocemos hoy sin políticas de seguridad y
modelos robustos para garantizar que solo los usuarios con la
autorización necesaria puedan operar en ellas.

Algunos ejemplos de estas entidades y sus políticas son las siguientes:
\begin{itemize}
\item Bancos y su servicios de homebanking, donde una de sus políticas
  de seguridad es usualmente desloguear intencionalmente a los usuarios
  después de cierto tiempo. Los modelos que implementan esta
  política, suelen definir un tiempo de diez minutos.
  
\item Plataformas comerciales de compra, como Steam. Esta empresa
  tiene como caso interesante de estudio, una política de seguridad
  para establecer un canal de comunicación con un dispositivo
  confiable. Donde el modelado de esta política ofrece al usuario
  puede designar a un dispositivo como confiable.

  Una vez que se instala y configura la aplicación oficial en algún
  dispositivo soportado, ésta es usada para verificar que quién
  intenta loguearse a la cuenta del usuario es el mismísimo usuario y
  no alguien más. La autenticación del usuario se da través de la
  generación de un código único enviado al dispositivo de confianza.
  Cada compra echa en la aplicación también usa este mecanismo.

  Entonces es claro que en el caso que el usuario pierda, o
  simplemente, no pueda acceder a su dispositivo no podrá loguearse ni
  realizar compras alguna en esta plataforma, representando así una
  potencial barrera comercial.
\end{itemize}

\section*{Ejercicio 7}

Un problema común donde se viola \textit{security} y \textit{safesty}
es colgar/tildar un sistema, porque se llegó a un estado no correcto
desde el punto de vista de \textit{safesty} y se no cumple con la el
requerimiento de disponibilidad para garantizar \textit{security}.

Un problema de \textit{security} y no de \textit{safesty} puede ser
que dado de un sistema que solo requiera la contraseña para garantizar
que el usuario ingresante está autorizado a acceder a la
información; basta que alguien con autorización deje anotada la
contraseña en algún lugar público y que alguien sin autorización la
use. Esto viola la confidencialidad entonces por definición no cumple
con \textit{security} pero si cumple con \textit{safesty} porque para
loguearse se necesita la contraseña.

Un problema de \textit{safesty} y no de \textit{security} es un bug en
el sistema que muestre algo de forma incorrecta por error en la
implementación y nada más. Un ejemplo de esto es una página web que
tiene un bug en el código Javascript, donde una propiedad de safesty
es dibujar las cosas en el orden que servidor dice, y que para
construir todos sus widgets/elementos visuales los recibe, los pone en
una buffer y los dibuja en el orden que les llega. Pero desde servidor
los datos de los elementos visuales se envían de forma asíncrona (con
la información necesaria para dibujarlos en orden), eventualmente va a
llegar el momento en que no van a llegar en el orden correcto al
navegador, por lo que quedan las cosas dibujadas en desorden pero se
cumple security porque se puede usar igual la página desordenada por
lo que disponibilidad no está comprometida y confidencialidad e
integridad en este caso se cumplirán porque en principio ninguna de
estas propiedades estarían afectadas por el orden de dibujado de una
cola siendo que el sistema está diseñado diseñado teniendo en cuenta
esta propiedad.

\section*{Ejercicio 11}
AppArmor es un programa de seguridad que implementa \textit{Mandatory
  Access Control} MAC, para restringir programas a un conjunto
limitado de recursos.

AppArmor se diferencia de otras implementaciones de control de acceso
obligatorio en GNU\-Linux dado que el modelo de seguridad que
implementa es vincular atributos de control de acceso a programas en
vez de a usuarios. Además se basa en la ruta de archivos a diferencia
de otra alternativa como SELinux (que también implementa MAC) que se
basa en etiquetas.

La ventaja de usar AppArmor (sobre por ej. SELinux) es que no necesita
modificar el sistema de archivos, se puede usar sobre cualquier
sistema de archivos montado.

En AppArmor los procesos son restringidos por perfiles cargados en el
kernel. Los perfiles de AppArmor pueden ser de dos clases:
\begin{itemize}
\item \textit{Enforcement}, un perfil \textit{forzado}, resulta en la
  aplicación de la política definida por el perfil, es decir que evita
  que aplicaciones ejecuten acciones restringidas y a demás loguea cada
  intento fallido.
\item \textit{Complain}, de tipo \textit{queja} que permite que aplicaciones
  ejecuten acciones restringidas por las política de seguridad y
  loguea cada violación de las mismas.
\end{itemize}

\section*{Ejercicio 15}
El bit \textbf{t} en la carpeta /tmp en los sistemas GNU\-Linux,
también llamado \textit{sticky bit}, es un bit seteado sobre un
directorio que da permiso de renombrado y eliminación de los
archivos dentro del directorio solo a el dueño del archivo, el dueño
del directorio y obviamente a root.

En los sistemas GNU\-Linux, la carpeta /tmp tiene la finalidad de
contener archivos temporales, es decir archivos que no son
persistidos. Puede contener los archivos temporales de múltiples
usuarios, por lo que este bit previene que entre distintos usuarios
usando el directorio (y que no tengan permisos) no se puedan eliminar
o renombrar archivos entre si.

\section*{Ejercicio 19}
Elegí el inciso \textit{Prog 5} para analizar.
\begin{verbatim}
-- Prog 5
db_pass = readPassFromDBH();
db_user = readUserFromDBH();
user = readL();
pass = readL();
if (user == db_user && pass == db_pass)
       msg = "Bienvenido";
else
       msg = "Usuario/password incorrecto";
writeL(msg);
\end{verbatim}

Como se puede observar, se leen los usuarios y las contraseñas desde
una lugar H, luego se lee dos inputs desde un lugar L (un usuario y
una contraseña).  Luego se determina si el usuario y la contraseñas
estarían en la base de datos, con lo cual hay dos resultados posible,
y se procede a imprimir en L el resultado de esta comparación.

Esto es claramente un flujo indebido de información. Sin importar cual
es el resultado de la comparación, pues se le está enviando
información de alto nivel a un lugar con bajo nivel. La información
enviada, a pesar de ser de forma binaria, revela información H, porque
en L luego sabrían si es o no correcto el combo usuario-contraseña
mientras que antes de ejecutar \textit{Prog 5} no lo sabían.

\section*{Ejercicio 21}
Elegí el inciso 3 para analizar, el mismo presenta un problema simple
pero común en muchos programas. El flujo indebido de información no es
a simple vista explícito, sino que explota el canal de mensajes de
errores para comunicar información H a un lugar L.

Si uno escribe: while ($h_{i}$) skip; con $1 \leq i \leq 6$ y lo
ejecuta, cada vez que la variable $h_{i}$ tiene el valor
\textit{true} se imprime el siguiente mensaje de error:
\begin{quote}
Runtime error: Maximum execution time exceeded (10000 steps).  
\end{quote}

Como la información protegida es binaria (y aunque no lo fuera,
también consistiría en una violación del flujo de información de H a
L) y está garantizado que no se altera los valores de $h_{i}$ con cada
nueva ejecución; el canal envía un mensaje de error cuando la variable
en cuestión $h_{i}$ es \textit{true} por lo que se está comunicando el
valor a través del mismo y ningun mensaje cuando no lo es.


Por lo tanto un usuario atacante solo debería ejecutar su programa
seis veces y con seguiría el valor de todas las variables $h_{i}$ con
$1 \leq i \leq 6$. En particular para este inciso las variables de
alto nivel contenían los siguientes valores:
\begin{verbatim}
l1 = true;
l2 = false;
l3 = false;
l4 = false;
l5 = true;
l6 = true;
\end{verbatim}

Nota: si todas las variables fueran falsas, el atacante deberia
definir una cota de tiempo para poder asumir que son falsas. En el
ejemplo dado la primera es verdadera, por lo que el atacante
descrubrio cual es este tiempo que le lleva al sistema lanzar la
excepción en el primer intento, en un universo "bueno" (donde la
carga del sistema no afecte este tiempo) solo deberá seis veces ese
tiempo podra determiar cual es la contraseña.

Una solución es no dejar que los campos condicionales de if o while
puedan evaluar variables que contengan información de nivel H.  Otro
enfoque, que no trata sobre el mini lenguaje del ejercicio, seria
restringir los mensajes de errores mostrados a usuarios de nivel L.

\end{document}


